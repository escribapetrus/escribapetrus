% Created 2023-06-15 Thu 16:49
% Intended LaTeX compiler: xelatex
\documentclass[11pt]{article}
\usepackage{graphicx}
\usepackage{longtable}
\usepackage{wrapfig}
\usepackage{rotating}
\usepackage[normalem]{ulem}
\usepackage{amsmath}
\usepackage{amssymb}
\usepackage{capt-of}
\usepackage{hyperref}
\author{Pedro Schreiber}
\date{2023-06-15}
\title{Pretty writing in Emacs: exporting org-mode files to PDF}
\hypersetup{
 pdfauthor={Pedro Schreiber},
 pdftitle={Pretty writing in Emacs: exporting org-mode files to PDF},
 pdfkeywords={},
 pdfsubject={},
 pdfcreator={Emacs 28.2 (Org mode 9.5.5)}, 
 pdflang={English}}
\begin{document}

\maketitle

\section*{Introduction: Org to \LaTeX{} to PDF}
\label{sec:orged82bc8}

There are two types of Emacs users: those who give up and go back to vscode, and those who get
full citizenship and move permanently to Emacs town. This article is for those masochists who don't care
if Google provides an easier tool for writing pretty texts and code notebooks.

\textbf{Org-mode} is an awesome tool for writing multi-purpose text in Emacs. It can be used for
\href{https://en.wikipedia.org/wiki/Literate\_programming}{literate programming} (like Jupyter Notebooks) in any languages we want, for todo-lists and schedules,
for articles and much more.

We have been using Org files for our own notetaking for a few months now, but until yesterday we didn't
really explore the \emph{export} and \emph{publish} functionalities, which allow us to transform org texts into
formats used in the world outside of Emacs, and package a collection of texts into a cohesive unit.
(we will not explore \emph{publish} in this article, as we have not yet used it).

We want to configure Emacs to export an Org text to a pretty PDF. Let's use this article, which we are
currently writing in org-mode (inception!) as an example. Our objective is to use \texttt{org-export-dispatch (C-c C-e)}
to \emph{export to \LaTeX{} as PDF file and open}.

\section*{\LaTeX{}: what is it, installation, exporting PDF}
\label{sec:org0e2f3d0}

The first step in our pipeline from Emacs Org to PDF is exporting our document to \LaTeX{}. Emacs can't do that
if the program is not installed. So, before we try to export our Org file, let us make sure that we can correctly
produce a PDF from \LaTeX{} code.

\subsection*{What is \LaTeX{}?}
\label{sec:orgdb3ca2c}

From the \href{https://www.latex-project.org/about/}{\LaTeX{} documentation}:

\begin{quote}
\LaTeX{} is a high-quality typesetting system; it includes features designed
for the production of technical and scientific documentation.
\LaTeX{} is the de facto standard for the communication and publication of scientific documents.
\end{quote}

In this sense, \LaTeX{} is a hierarchical text definition language, like markup or html, which allow us to describe
the role of each section in the context of the whole text. This is an example of \LaTeX{} code, taken from the docs.

\begin{verbatim}
\documentclass{article}
\title{Cartesian closed categories and the price of eggs}
\author{Jane Doe}
\date{September 1994}
\begin{document}
   \maketitle
   Hello world!
\end{document}
\end{verbatim}

The \texttt{documentclass} section describes the template for the output document; in other words, the \emph{style},
which is defined in the \LaTeX{} document template.

In summary, \LaTeX{} is a language and a program for:
\begin{itemize}
\item hierarchical text definition language (like markup, html)
\item format and style definition language (like css)
\item text processing program
\end{itemize}

\subsection*{Installation}
\label{sec:org7195fd2}

If we open the \href{https://www.latex-project.org/get/}{\LaTeX{} Project GET page}, we will find the recommended distributions for the most popular
operating systems. We are configuring this on our MacBook Air, so we should get MacTeX. However, the complete MacTeX bundle is 5GB,
which is absurdly huge for our humble project. Rather, let us go for the \emph{much smaller} 90MB \href{https://www.tug.org/mactex/morepackages.html}{BasicTeX}.

The BasicTeX .pkg file is very much plug and play, as it installs the binaries to folders already supported in the terminal PATH.
For generating a PDF from \LaTeX{} code, will use XeLaTeX, an improved version of \LaTeX{} with support for unicode and special fonts.

\subsection*{Fixing problems and exporting the PDF}
\label{sec:org1432b45}

Even the minimal BasicTeX comes with the necessary tools to generate a PDF. Therefore, we should be ready to export
our PDF after the installation. Let's save the the \LaTeX{} code example provided earlier as \texttt{article.tex}
and try exporting it to PDF by using the \texttt{xelatex article.tex} command.

In our machine we have encountered some problems, registered in a .log file. Let us see what they are and
how to solve them.

\begin{verbatim}
$ xelatex article.tex
...
(/usr/local/texlive/2023basic/texmf-dist/tex/latex/amsmath/amsopn.sty))
(/usr/local/texlive/2023basic/texmf-dist/tex/latex/amsfonts/amssymb.sty
 (/usr/local/texlive/2023basic/texmf-dist/tex/latex/amsfonts/amsfonts.sty))

! LaTeX Error: File 'wrapfig.sty' not found.
\end{verbatim}

The error is very clear. \LaTeX{} cannot find the \href{https://www.ctan.org/tex-archive/macros/latex/contrib/wrapfig}{wrapfig package}. Our distribution must have come without it.
The good thing is that the error message even shows us where to put it. The fix is easy, we just download the file
and store it where the error pointed us, maybe in the \texttt{graphics/} folder:

\begin{verbatim}
$ wget http://mirrors.ctan.org/macros/latex/contrib/wrapfig/wrapfig.sty
$ sudo mv wrapfig.sty /usr/local/texlive/2023basic/texmf-dist/tex/latex/graphics/wrapfig.sty
\end{verbatim}

If we run \texttt{xelatex article.tex} again, we get a different error, and that is progress; it means the first problem
was solved. Here is the second error:

\begin{verbatim}
$ xelatex article.tex
...
(/usr/local/texlive/2023basic/texmf-dist/tex/latex/amsmath/amsopn.sty))
(/usr/local/texlive/2023basic/texmf-dist/tex/latex/amsfonts/amssymb.sty
(/usr/local/texlive/2023basic/texmf-dist/tex/latex/amsfonts/amsfonts.sty))

! LaTeX Error: File 'capt-of.sty' not found.
\end{verbatim}

This is almost the same error as before. \LaTeX{} cannot find the \href{https://www.ctan.org/tex-archive/macros/latex/contrib/capt-of}{capt-of} package. The solution here is a bit more challenging,
because the source does not provide the .sty file (don't ask us why). We have to compile it ourselves.

\begin{verbatim}
$ wget http://mirrors.ctan.org/macros/latex/contrib/capt-of/capt-of.dtx
$ wget http://mirrors.ctan.org/macros/latex/contrib/capt-of/capt-of.ins

$ xelatex capt-of.ins

$ sudo mv capt-of.sty /usr/local/texlive/2023basic/texmf-dist/tex/latex/caption/capt-of.sty
\end{verbatim}

Now, if we run \texttt{xelatex article.tex}, we get a beautiful \texttt{article.pdf} exported by \LaTeX{}. Awesome. 

\section*{Running on Emacs}
\label{sec:orgcecf917}

The essential part is basically done. All we have to do now in order to export our org files to PDF
is telling Emacs what \LaTeX{} compiler and command to use. For that, just add to the elisp configuration file
(e.g. \texttt{init.el}):

\begin{verbatim}
(setq org-latex-compiler "xelatex")
(setq org-latex-pdf-process '("xelatex %f"))
\end{verbatim}

We can now run \texttt{M-x org-export-dispatch (C-c C-e)} to \emph{export to \LaTeX{} as PDF file and open} (options l-o),
just like we wanted. In fact, we have done it in just the org file for this article, and it worked beautifully.

This concludes our explanation of how to export an Org text to a pretty PDF. One thing to add is that we are not
academics who produce papers frequently, so our knowledge of \LaTeX{} is \emph{very} limited. Feel free to leave any suggestions
and corrections in the commentaries.
\end{document}